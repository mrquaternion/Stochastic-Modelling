\documentclass{article}
\usepackage[utf8]{inputenc}
\usepackage{amsmath}
\usepackage{amssymb}
\usepackage{graphicx}
\usepackage{enumitem}
\usepackage{tikz}
\usepackage[outputdir=build]{minted}
\usepackage{fancyvrb}

\RecustomVerbatimCommand{\VerbatimInput}{VerbatimInput}%
{fontsize=\footnotesize,
 frame=lines,  
 framesep=2em, 
 rulecolor=\color{gray},
 label=\fbox{\color{black}output},
 labelposition=topline,
 commandchars=\|\(\), % escape character and argument delimiters for
                      % commands within the verbatim
 commentchar=*        % comment character
}

\usetikzlibrary{shapes, arrows, positioning}

\title{\textbf{DEVOIR 4}}
\author{Mathias La Rochelle}
\date{Le jeudi 10 octobre 2024}

\begin{document}
\maketitle

%%%%%%%%%%%%%%%%%%%%%%%%%%%%%%%%%%%%%%%%%%%%%%%%%%%%%%%%%%%%%%%%%%%%%%%%%%%%%%%%%%%%%%%%
\section*{Problème 1}
\textbf{Énoncé}. L'état d'un processus de fabrication change chaque jour 
selon une chaîne de Markov à deux états, 0 et 1, avec les probabilités de
transition suivantes :
\[
    P_{0,0} = 0.4,\hspace{.3cm} P_{0,1} = 0.6,\hspace{.3cm} P_{1,0} = 0.2,\hspace{.3cm} P_{1,1} = 0.8
\]
Chaque jour, un test est fait sur le processus et si la chaîne est dans
l'état $i$, le test retourne ``bon" avec probabilité $p_i$ et ``mauvais" avec
probabilité $q_i=1-p_i$, pour $i=0,1$. Vos réponses seront des expressions
qui dépendent de $p_0$ et $p_1$. On peut noter $X_n$ l'état de la chaîne 
au jour $n$, puis poser $Y_n=1$ si le test retourn ``bon" au jour $n$, et 
$Y_n=2$ sinon. \textit{Le vrai état du processus n'est pas observé, 
seulement $Y_n$ est observé.}
\vspace{.6cm}
\hrule
\vspace{.2cm}
\subsection*{Partie A}
\textbf{Énoncé}. Si le processus est dans l'état 0 un jour donné, quelle
est la probabilité que le test retourne ``bon" le lendemain ?
\vspace{.2cm}
\hrule
\vspace{.4cm}
La probabilité que le test retourne ``bon" le lendemain sachant que le 
processus est à l'état 0 au jour donné signifie qu'on cherche $Y_n=1$ 
quand $n=1$ soit
\begin{align}
    \mathbb{P}[Y_1=1\mid X_0=0]&=\mathbb{P}[Y_1=1\mid X_1=0,\,X_0=0]\cdot\mathbb{P}[X_1=0,\,X_0=0] \notag \\
    &+\mathbb{P}[Y_1=1\mid X_1=1,\,X_0=0]\cdot\mathbb{P}[X_1=1,\,X_0=0] \notag \\
    &=p_0\cdot 0.4+p_1\cdot 0.6 \notag \\
    &=0.4p_0+0.6p_1 \notag 
\end{align}
La probabilité que le test retourne ``bon" à ce jour-là est de $0.4p_0+0.6p_1$.

\newpage
\subsection*{Partie B}
\textbf{Énoncé}. Si le processus est dans l'état 0 le lundi, quelle est
la probabilité que le test retourne ``bon" le vendredi suivant ?
\vspace{.2cm}
\hrule
\vspace{.4cm}
La probabilité que le test retourne ``bon" le vendredi suivant sachant
que le processus est à l'état 0 le lundi signifie qu'on cherche $Y_n=1$ 
quand $n=4$ soit
\begin{align}
    \mathbb{P}[Y_4=1\mid X_0=0] &= \mathbb{P}[Y_4=1\mid X_4=0]\cdot\mathbb{P}[X_4=0\mid X_0=0] \notag \\
    &+ \mathbb{P}[Y_4=1\mid X_4=1]\cdot\mathbb{P}[X_4=1\mid X_0=0]
\end{align}
Pour pouvoir trouver de se rendre de l'état 0 à l'état $i$ en $n=4$ étapes, 
nous allons utiliser l'équation de Chapman-Kolmogorov.
\[
    P^{(n)}_{i,j}=\mathbb{P}[X_n=j\mid X_0=i]
\]
On a les deux probabilités suivantes
\begin{gather}
    P^4_{0,0}=\mathbb{P}[X_4=0\mid X_0=0] \notag \\
    P^4_{0,1}=\mathbb{P}[X_4=1\mid X_0=0] \notag
\end{gather}
qui seront calculées à partir de la matrice $\boldsymbol{P}^4$ :
\begin{align}
    \boldsymbol{P}^4&=\boldsymbol{P}^2\times\boldsymbol{P}^2 \notag \\
    &= \left(\boldsymbol{P}\times\boldsymbol{P}\right)\times\boldsymbol{P}^2 \notag \\
    &= \left(\begin{pmatrix} 0.4 & 0.6 \\ 0.2 & 0.8 \end{pmatrix}\times\begin{pmatrix} 0.4 & 0.6 \\ 0.2 & 0.8 \end{pmatrix}\right)\times\boldsymbol{P}^2 \notag \\
    &= \begin{pmatrix} 0.28 & 0.72 \\ 0.24 & 0.76 \end{pmatrix}\times\boldsymbol{P}^2 \notag \\
    &= \begin{pmatrix} 0.28 & 0.72 \\ 0.24 & 0.76 \end{pmatrix}\times\begin{pmatrix} 0.28 & 0.72 \\ 0.24 & 0.76 \end{pmatrix} \notag \\
    &= \begin{pmatrix} 0.2512 & 0.7488 \\ 0.2496 & 0.7504 \end{pmatrix} \notag
\end{align}
On substitue les probabilités de cette matrice à notre équation (1) :
\begin{align}
    \boldsymbol{P}[Y_4=1\mid X_0=0]&=p_0\cdot0.2512+p_1\cdot0.7488 \notag \\
    &=0.2512p_0+0.7488p_1 \notag
\end{align}
La probabilité que le test retourne ``bon" à ce jour-là est de $0.2512p_0+0.7488p_1$.

\newpage
\subsection*{Partie C}
\textbf{Énoncé}. Quelle est la proportion des jours où le test retourne
``bon" à long terme ?
\vspace{.2cm}
\hrule
\vspace{.4cm}
Pour trouver cette proportion à long terme, il faut d'abord trouver les 
probabilités des états stationnaires à long terme. On résoud les équations
d'équilibre $\boldsymbol{\pi}=\boldsymbol{\pi}\boldsymbol{P}$ et
$\boldsymbol{\pi}\boldsymbol{1}^t=1$ :
\begin{align}
    \pi_0&=0.4\pi_0+0.2\pi_1 \\
    \pi_1&=0.6\pi_0+0.8\pi_1 \\
    1&=\pi_0+\pi_1
\end{align}
Avec (2) et (4) on a
\begin{align}
    \pi_0&=0.4\pi_0+0.2(1-\pi_0) \notag \\
    0.6\pi_0&=0.2-0.2\pi_0 \notag \\
    0.8\pi_0&=0.2 \notag 
\end{align}
qui donne
\[
    \pi_0=\frac{1}{4},\ \pi_1=1-\frac{1}{4}=\frac{3}{4} 
\]

\vspace{.3cm}
En conclusion, la proportion des jours où le test retourne ``bon" à long 
terme dépend de $p_0$ et de $p_1$ et est donnée par l'équation ci-dessous.
\begin{align}
    \mathbb{P}[\text{Le test retourne ``bon" à long terme}]&=\pi_0p_0+\pi_1p_1 \notag \\
    &=0.25p_0+0.75p_1 \notag
\end{align}

\newpage
\subsection*{Partie D}
\textbf{Énoncé}. Est-ce que $\{Y_n,\,n\ge0\}$ est une chaîne de Markov ?
Si oui, donnez sa matrice de probabilités de transition; sinon expliquez
pourquoi.
\vspace{.2cm}
\hrule
\vspace{.4cm}
\noindent \textbf{Clarification} : $Y_n$ est l'état du test au jour $n$, c'est-à-dire qu'au $n$-ième jour, 
le test peut soit être ``bon" ou `mauvais". \\

\noindent \textbf{Dépendance de} $Y_{n+1}$ : \\
Le résultat du test $Y_{n+1}$ dépend de l'état du processus $X_{n+1}$ et des 
probabilités associées à cet état ($p_0$ ou $p_1$). Cependant, le passage de 
$X_n$ à $X_{n+1}$ est une chaîne de Markov, mais $Y_{n+1}$ ne dépend pas 
directement de $Y_n$. $Y_n$ est plutôt une observation indirecte qui dépend 
de l'état ``caché" $X_n$. \\

Puisque $Y_{n+1}$ dépend de l'état caché $X_{n+1}$ et non directement de $Y_n$, 
la séquence $\{Y_n,\, n \geq 0\}$ ne satisfait pas la propriété markovienne. Elle
ne constitue donc pas une chaîne de Markov par elle-même. Les transitions dans 
$\{Y_n\}$ sont influencées par les états cachés $X_n$ ce qui, au final, empêche 
$\{Y_n\}$ d'être une chaîne de Markov indépendante.



%%%%%%%%%%%%%%%%%%%%%%%%%%%%%%%%%%%%%%%%%%%%%%%%%%%%%%%%%%%%%%%%%%%%%%%%%%%%%%%%%%%%%%%%
\newpage
\section*{Problème 2}
\textbf{Énoncé}. Voir mes diapos, pages 52-53.
\vspace{.6cm}
\hrule
\vspace{.2cm}
\subsection*{Partie A}
\text{Énoncé}. Dans l'exemple de la marche aléatoire sur $\{0,1,...,N\}$, 
quand les états 0 et $N$ sont absorbants et les $p_i$ peuvent dépendre de
$i$, dérivez une formule pour $P_1$ et une formule pour chaque $P_i$ en
fonction de $P_1$.
\vspace{.2cm}
\hrule
\vspace{.4cm}
Toute probabilité $P_i$ représente la probabilité de se rendre à $P_N$
en partant de l'état $i$. De soit, si on tombe dans l'état 0, on ne peut
aller vers l'état $N$ donc, $P_0=0$. Inversement, si on tombe dans l'état
$N$, alors on est sûr d'y rester donc, $P_N=1$. Ces cas spéciaux proviennent
du fait que ces deux états sont absorbants.

\vspace{.3cm}
Le graphe de la chaîne de Markov qui décrit ce problème est
\begin{figure}[h]
    \centering
    \begin{tikzpicture}[->,>=stealth',shorten >= 2pt, line width=0.5pt, node distance=2cm]
        \node [circle, draw] (zero) at (0, 0) {0};
        \path (zero) edge [loop left] node {$1$} (zero);
        \node [circle, draw] (one) [right of=zero] {1};
        \path (one) edge [bend left] node [below] {$q_1$} (zero);
        \node (dots1) [right of=one] {...};
        \path (one) edge [bend left] node [above] {$p_1$} (dots1);
        \path (dots1) edge [bend left] node [below] {$q_2$} (one);
        \node [circle, draw] (i) [right of=dots1] {i};
        \path (dots1) edge [bend left] node [above] {$p_{i-1}$} (i);
        \path (i) edge [bend left] node [below] {$q_i$} (dots1);
        \node (dots2) [right of=i] {...};
        \path (i) edge [bend left] node [above] {$p_i$} (dots2);
        \path (dots2) edge [bend left] node [below] {$q_{i+1}$} (i);
        \node [circle, draw] (n) [right of=dots2] {N};
        \path (dots2) edge [bend left] node [above] {$p_{n-1}$} (n);
        \path (n) edge [loop right] node {$1$} (n);
    \end{tikzpicture}
    \caption{Marche aléatoire sur $\{0,1,...,N\}$}
\end{figure}

\noindent On en dérive la formule suivante :
\[
    P_i=p_i\cdot P_{i+1}+q_i\cdot P_{i-1},\ \text{où } 0<i<N 
\]
qui peut être réecrite comme
\begin{gather}
    p_i\cdot P_i+q_i\cdot P_i=p_i\cdot P_{i+1}+q_i\cdot P_{i-1},\ \text{parce que } p_i+q_i=1 \notag \\
    p\cdot P_i-p_i\cdot P_{i+1}=q_i\cdot P_{i-1}-q_i\cdot P_i \notag \\
    P_i-P_{i+1}=\frac{q_i}{p_i}\left(P_{i-1}-P_i\right)
\end{gather}
On peut désormais utiliser l'équation (5) pour trouver les $P_i,\,i\in\{2,3,...,N-1\}$ 
en fonction de $P_1$. Avec $i=1$, on a
\begin{align}
    P_1-P_2&=\frac{q_1}{p_1}\left(P_0-P_1\right) \notag \\
    P_1-P_2&=\frac{q_1}{p_1}\left(-P_1\right),\ \text{étant donné que } P_0=0 \notag \\
    P_2&=P_1\left(1+\frac{q_1}{p_1}\right) \notag
\end{align}
Subséquemment, avec $i=2$ on a
\[
    P_2-P_3=\frac{q_2}{p_2}(P_1-P_2)=\frac{q_2}{p_2}\left(P_1-P_1\left(1+\frac{q_1}{p_1}\right)\right)=-P_1\frac{q_2\cdot q_1}{p_2\cdot p_1}
\]
Puis, avec $i=3$ on a
\[
    P_3-P_4=\frac{q_3}{p_3}(P_2-P_3)=-P_1\frac{q_3\cdot q_2\cdot q_1}{p_3\cdot p_2\cdot p_1}
\]
En suivant cette suite récurrente, on retrouve la formule pour les $P_i$ :
\begin{equation}
    P_i-P_{i+1}=-P_1\cdot \prod_{j=1}^{i}\frac{q_j}{p_j}
\end{equation}
Pour simplifier l'équation et trouver celle de $P_1$, on somme les différences :
\[
    \sum_{m=1}^{N-1}(P_m-P_{m+1})=(P_1-P_2)+(P_2-P_3)+\dots+(P_{N-1}-P_N)=P_1-P_N
\]
\vspace{-.7cm}
\begin{figure}[h]
    \begin{align}
        P_1-P_N&=\sum_{m=1}^{N-1}\left(-P_1\cdot \prod_{j=1}^{m}\frac{q_j}{p_j}\right) \notag \\
        P_1&=1+\sum_{m=1}^{N-1}\left(-P_1\cdot \prod_{j=1}^{m}\frac{q_j}{p_j}\right),\ \text{avec } P_N=1\ \ ^* \notag \\
        -1&=\sum_{m=1}^{N-1}\left(\prod_{j=1}^{m}\frac{q_j}{p_j}\right)-\frac{1}{P_1} \notag \\
        \frac{1}{P_1}&=1+\sum_{m=1}^{N-1}\left(\prod_{j=1}^{m}\frac{q_j}{p_j}\right) \notag \\
        P_1&=\left[1+\sum_{m=1}^{N-1}\left(\prod_{j=1}^{m}\frac{q_j}{p_j}\right)\right]^{-1}
    \end{align}
    \caption{* : parce que une fois dans l'état $N$, on est assuré d'y rester}
\end{figure}

\noindent On peut faire le même processus pour les $P_i$ :
\[
    \sum_{m=1}^{i-1}(P_m-P_{m+1})=(P_1-P_2)+(P_2-P_3)+\dots+(P_{i-1}-P_i)=P_1-P_i
\]
Puis,
\begin{align}
    P_1-P_i&=\sum_{m=1}^{i-1}\left(-P_1\cdot \prod_{j=1}^{m}\frac{q_j}{p_j}\right) \notag \\
    P_i&=P_1-\sum_{m=1}^{i-1}\left(-P_1\cdot \prod_{j=1}^{m}\frac{q_j}{p_j}\right) \notag \\
    P_i&=P_1\left(1+\sum_{m=1}^{i-1}\left(\prod_{j=1}^{m}\frac{q_j}{p_j}\right)\right)
\end{align}

Les équations (7) et (8) sont les formules respectives des probabilités de se
finir dans l'état absorbant $N$ en débutant par l'état 1 et l'état $i$.

\newpage
\subsection*{Partie B}
\text{Énoncé}. Si $N=5$ et $p_i=(N-i)/N$ pour $0<i<N$, calculez les valeurs des $P_i$.
\vspace{.2cm}
\hrule
\vspace{.4cm}
Déterminons tout d'abord les $p_i,\, i\in\{1,2,3,4\}$ :
\begin{gather}
    p_1=\frac{5-1}{5}=\frac{4}{5},\ \text{avec } i=1 \\
    p_2=\frac{5-2}{5}=\frac{3}{5},\ \text{avec } i=2 \\
    p_3=\frac{5-3}{5}=\frac{2}{5},\ \text{avec } i=3 \\
    p_4=\frac{5-4}{5}=\frac{1}{5},\ \text{avec } i=4
\end{gather}
qui nous permettent de calculer les $P_i$ suivants en Python
\begin{minted}[bgcolor=lightgray, fontsize=\footnotesize, linenos]{python}
N = 5
probabilites = [(N - i) / N for i in range(1, N)]

somme_int = 0
for m in range(1, N): # s'arrête à N - 1
    produit_int = 1
    for j in range(1, m + 1): # s'arrête à m
        produit_int *= (1 - probabilites[j - 1]) / probabilites[j - 1]

    somme_int += produit_int

P_1 = 1 / (1 + somme_int)
print("La probabilité de se rendre à l'état N en débutant de l'état 1 est de "
       + (round(P_1, 3) * 100) + "%.")

for i in range(2, N):
    somme_int = 0
    for m in range(1, i): # s'arrête à i - 1
        produit_int = 1
        for j in range(1, m + 1):
            produit_int *= (1 - probabilites[j - 1]) / probabilites[j - 1]

        somme_int += produit_int

    P_i = P_1 * (1 + somme_int)
    print("La probabilité de se rendre à l'état N en débutant de l'état " + i 
           + " est de " + (round(P_i, 3) * 100) + "%.")
\end{minted}

\VerbatimInput{output.txt}

%%%%%%%%%%%%%%%%%%%%%%%%%%%%%%%%%%%%%%%%%%%%%%%%%%%%%%%%%%%%%%%%%%%%%%%%%%%%%%%%%%%%%%%%
\newpage
\section*{Problème 3}
\textbf{Énoncé}. Soit $a$ et $b$ deux entiers positifs et considérons une chaîne de 
Markov $\{\mathbf{X}_n,\,n\ge0\}$ dont l'espace d'états est 
$\{(x,y):x\in\{0,...,a-1\},\,y\in\{0,...,b-1\}\}$, et les probabilités de transition 
sont définies comme suit. Si $\mathbf{X}_n=(x,y)$, alors 
$\mathbf{X}_{n+1}=((x+1)\hspace{-.1cm}\mod{a},y)$ ou 
$\mathbf{X}_{n+1}=(x,(y+1)\hspace{-.1cm}\mod{b})$ chacun avec probabilité 1/2.
\vspace{.6cm}
\hrule
\vspace{.2cm}
\subsection*{Partie A}
\textbf{Énoncé}. Montrez que cette chaîne est irréductible.
\vspace{.2cm}
\hrule
\vspace{.4cm}
Pour montrer que la chaîne de Markov décrite est irréductible, nous devons prouver 
que tout état dans l'espace d'états $\{(x,y):x\in\{0,\dots,a-1\},\,y\in\{0,\dots,b-1\}\}$ 
peut être atteint à partir de n'importe quel autre état, en un certain nombre d'étapes. \\

Les états de cette chaîne sont de la forme $(x, y)$, où $x$ varie de $0$ à $a-1$ et $y$ de $0$ à $b-1$. Les transitions possibles sont :
\begin{itemize}
    \item De $(x, y)$ à $((x+1)\hspace{-.1cm}\mod a, y)$ avec probabilité $1/2$
    \item De $(x, y)$ à $(x, (y+1)\hspace{-.1cm}\mod b)$ avec probabilité $1/2$
\end{itemize}

Ces transitions signifient qu'à chaque étape, soit la première composante $x$ 
est incrémentée de $1\hspace{-.1cm}\mod a$, soit la deuxième composante $y$ est incrémentée 
de $1\hspace{-.1cm}\mod b$ et ce, avec probabilité égale. \\

Désormais, supposons que nous partions d'un état $(x,y)$ et que nous souhaitions 
atteindre un autre état $(x^\prime,y^\prime)$. 
\begin{itemize}
    \item Pour la première composante $x$, en appliquant la transition 
    $x \to (x+1)\mod a$ de façon répétée, nous pouvons passer à n'importe quelle 
    autre valeur de $x$ en au plus $a$ étapes, car le compteur $x$ se comporte 
    comme une roue cyclique avec $a$ positions.
    \item De manière similaire, pour la deuxième composante $y$, en appliquant la 
    transition $y \to (y+1)\mod b$, nous pouvons atteindre n'importe quelle autre 
    valeur de $y$ en au plus $b$ étapes, puisque $y$ se comporte comme une roue 
    cyclique avec $b$ positions.
\end{itemize}
Ainsi, il est toujours possible de trouver un chemin dans l'espace d'états de 
$(x,y)$ à $(x^\prime,y^\prime)$ en combinant ces transitions. Cela implique qu'il 
existe un nombre fini d'étapes pour aller de n'importe quel état à n'importe 
quel autre. \\

Finalement, étant donné que tous les états communiquent entre eux, cela signifie 
que tous les états appartiennent à la même classe de communication. Cette classe est
\[
    C = \{(x,y):x\in\{0,\dots,a-1\},\,y\in\{0,\dots,b-1\}\}
\]
Donc, la chaîne de Markov est irréductible.

\newpage
\subsection*{Partie B}
\text{Énoncé}. Montrez qu'elle est apériodique si et seulement si pgcd$(a,b)=1$.
\vspace{.2cm}
\hrule
\vspace{.4cm}
Une chaîne de Markov est dite apériodique si tous les états accessibles sont
apériodiques. Un état $i$ est apériodique si et seulement si sa période $d(i)=1$. \\

\noindent Les transitions dans cette chaîne suivent deux cycles indépendants : un cycle 
de longueur \(a\) pour la composante \(x\), et un cycle de longueur \(b\) pour la composante 
\(y\). Quand \( \text{pgcd}(a, b) = 1 \), cela signifie qu'il n'existe pas de synchronisation 
stricte entre les cycles des deux composantes. Inversement, lorsque \( \text{pgcd}(a, b) > 1 \), 
les deux cycles \(a\) et \(b\) partagent un diviseur commun qui crée une synchronisation 
des transitions. Ainsi, les états de la chaîne se répètent périodiquement à des intervalles 
fixes.


\subsubsection*{Exemple 1}
Dans le cas où $a=1$ et $b=2$, on a l'espace d'états $\{\{x,y\}:x=0,\ y\in\{0,1\}\}$
et pgcd$(a,b)=1$. Les transitions entre les états de l'étape $n$ à $n+1$ sont :
\begin{itemize}
    \item $(x_n=0,\,y_n=0)\to(x_{n+1}=1\hspace{-.1cm}\mod 1,\,y_{n+1}=0)=(x_{n+1}=0,\,y_{n+1}=0)$
    \item $(x_n=0,\,y_n=0)\to(x_{n+1}=0,\,y_{n+1}=1\hspace{-.1cm}\mod 2)=(x_{n+1}=0,\,y_{n+1}=1)$
    \item $(x_n=0,\,y_n=1)\to(x_{n+1}=1\hspace{-.1cm}\mod 1,\,y_{n+1}=1)=(x_{n+1}=0,\,y_{n+1}=1)$
    \item $(x_n=0,\,y_n=1)\to(x_{n+1}=0,\,y_{n+1}=2\hspace{-.1cm}\mod 2)=(x_{n+1}=0,\,y_{n+1}=0)$
\end{itemize}
\begin{figure}[h]
    \centering
    \begin{tikzpicture}[->,>=stealth',shorten >= 2pt, line width=0.5pt, node distance=2cm]
        \node [circle, draw] (zero-zero) at (0, 0) {(0,0)};
        \node [circle, draw] (zero-one) at (3,0) {(0,1)};
        \path (zero-zero) edge [bend left] (zero-one);
        \path (zero-one) edge [bend left] (zero-zero);
        \path (zero-zero) edge [loop left] (zero-zero);
        \path (zero-one) edge [loop right] (zero-one);
    \end{tikzpicture}
    \caption{Diagramme des transitions entre les états (sans les probabilités)}
\end{figure}

On remarque qu'un cycle dans cette chaîne n'a pas besoin de suivre une certaine 
période fixe : chaque état est accessible à partir de lui-même en une étape (boucle), 
ou bien par alternance entre deux états $(0,0)$ et $(0,1)$. Comme la période des 
transitions entre ces états est $1$, cela montre que la chaîne est apériodique.

\subsubsection*{Exemple 2}
Dans le cas où $a=b=2$, on a l'espace d'états $\{\{x,y\}:x\in\{0,1\},\ y\in\{0,1\}\}$
et pgcd$(a,b)=2$. Les transitions entre les états de l'étape $n$ à $n+1$ sont :
\begin{itemize}
    \item $(x_n=0,\,y_n=0)\to(x_{n+1}=1\hspace{-.1cm}\mod 2,\,y_{n+1}=0)=(x_{n+1}=1,\,y_{n+1}=0)$
    \item $(x_n=0,\,y_n=0)\to(x_{n+1}=0,\,y_{n+1}=1\hspace{-.1cm}\mod 2)=(x_{n+1}=0,\,y_{n+1}=1)$
    \item $(x_n=1,\,y_n=0)\to(x_{n+1}=2\hspace{-.1cm}\mod 2,\,y_{n+1}=0)=(x_{n+1}=0,\,y_{n+1}=0)$
    \item $(x_n=1,\,y_n=0)\to(x_{n+1}=1,\,y_{n+1}=2\hspace{-.1cm}\mod 2)=(x_{n+1}=1,\,y_{n+1}=1)$
    \item $(x_n=0,\,y_n=1)\to(x_{n+1}=1,\,y_{n+1}=1)$ par symétrie
    \item $(x_n=0,\,y_n=1)\to(x_{n+1}=0,\,y_{n+1}=0)$ par symétrie
    \item $(x_n=1,\,y_n=1)\to(x_{n+1}=0,\,y_{n+1}=1)$ par symétrie
    \item $(x_n=1,\,y_n=1)\to(x_{n+1}=1,\,y_{n+1}=0)$ par symétrie
\end{itemize}
\begin{figure}[h]
    \centering
    \begin{tikzpicture}[->,>=stealth',shorten >= 2pt, line width=0.5pt, node distance=2cm]
        \node [circle, draw] (zero-zero) at (0, 2) {(0,0)};
        \node [circle, draw] (zero-one) at (0, 0) {(0,1)};
        \node [circle, draw] (one-zero) at (2, 2) {(1,0)};
        \node [circle, draw] (one-one) at (2, 0) {(1,1)};
        \path (zero-zero) edge [bend right] (zero-one);
        \path (zero-one) edge [bend right] (zero-zero);
        \path (zero-zero) edge [bend left] (one-zero);
        \path (one-zero) edge [bend left] (zero-zero);
        \path (one-zero) edge [bend left] (one-one);
        \path (one-one) edge [bend left] (one-zero);
        \path (zero-one) edge [bend right] (one-one);
        \path (one-one) edge [bend right] (zero-one);
    \end{tikzpicture}
    \caption{Diagramme des transitions entre les états (sans les probabilités)}
\end{figure}

Les transitions possibles entre les états $(x,y)$ permettent une alternance systématique 
entre les quatre états. Chaque état est atteint après un nombre d'étapes fixe, ce qui caractérise 
une chaîne périodique. Ils forment un cycle, et en raison du fait que les périodes pour chaque 
composante sont égales à 2, la chaîne est périodique avec une période $d(i)=2$. Chaque état 
$(x,y)$ peut retourner à lui-même ou aux autres états après un nombre d'étapes multiple de 2, 
ce qui rend cette chaîne non apériodique. \\

\noindent On conclut qu'une chaîne de Markov irréductible est apériodique si et seulement si 
pgcd$(a,b)=1$. Dans ce cas, les cycles des deux composantes $x$ et $y$ ne se synchronisent pas, 
et chaque état peut être atteint à des étapes non multiples d'une période fixe. Lorsque
pgcd$(a,b)>1$, les deux cycles se synchronisent, entraînant une périodicité commune à tous 
les états, rendant la chaîne périodique.


%%%%%%%%%%%%%%%%%%%%%%%%%%%%%%%%%%%%%%%%%%%%%%%%%%%%%%%%%%%%%%%%%%%%%%%%%%%%%%%%%%%%%%%%
\newpage
\section*{Problème 4}
\textbf{Énoncé}. On considère une suite de polygones convexes définis comme suit. 
À l'étape $n$, on a un polygone de $X_n+3$ côtés, pour $X_n\ge0$. Par exemple,
pour un triangle on a $X_n=0$ et pour un hexagone on a $X_n=3$. On choisit au hasard 
(uniformément) deux des $X_n+3$ côtés (faces) de ce polygone et on rejoint les points
milieux de ces deux faces par un segment de droite. Cela divise le polygone courant
en deux polygones convexes plus petits, $P_1$ et $P_2$. On suppose que le polygone 
initial n'a pas deux faces successives parfaitement alignées. Puisque c'est seulement
le nombre de faces qui compte, on peut se faire une image de la situation en supposant
que le polygone est régulier, à chaque étape. On choisit un des deux polygones $P_1$
ou $P_2$ au hasard, avec probabilité 1/2 chacun. et c'est le prochain polygone, dont
le nombre de faces est $X_{n+1}+3$.

\vspace{.3cm}
\noindent Par exemple, si on a un pentagone et qu'on choisit deux faces adjacents, 
il sera divisé en un triangle et un hexagone. Si on choisit deux faces non adjacentes, 
les deux sous-polygones obtenus auront 4 et 5 faces. Suggestion : faites-vous un 
dessin, énumérez tous les choix possibles, et calculez la probabilité de chaque choix.
\vspace{.6cm}
\hrule
\vspace{.2cm}
\subsection*{Partie A}
\textbf{Énoncé}. Montrez que $\{X_n,\,n\ge0\}$ est une chaîne de Markov irréductible.
Dites quel est son espace d'états et trouvez la matrice des probabilités de transition.
\vspace{.2cm}
\hrule
\vspace{.4cm}
L’espace d’états de la chaîne de Markov \(\{X_n,\,n\geq0\}\) est l’ensemble des nombres naturels \(\mathbb{N} = \{0, 1, 2, \dots\}\). Chaque état \(X_n = i\) représente le nombre de côtés supplémentaires par rapport au triangle initial, autrement dit, l'état \(i\) correspond à un polygone ayant \(i + 3\) côtés. Par exemple :
\begin{itemize}
    \item \(X_n = 0\) correspond à un polygone à 3 côtés (un triangle),
    \item \(X_n = 1\) correspond à un polygone à 4 côtés (un quadrilatère),
    \item \(X_n = 2\) correspond à un polygone à 5 côtés (un pentagone), et ainsi de suite.
\end{itemize}

On a donc l’espace d’états $E$ de la chaîne de Markov suivante :
\[
    E = \{0, 1, 2, 3, \dots\}
\]

Pour la suite des choses, afin de mieux comprendre notre chaîne de Markov, nous écrivons 
le processus $Y_n$ comme étant le nombre de côtés que possède un polygone à l'étape $n$ 
soit $Y_n=X_n+3$.

\newpage
\begin{figure}[h]
    \centering
    \resizebox{0.6\textwidth}{!}{\tikzset{every picture/.style={line width=0.75pt}} %set default line width to 0.75pt        

\begin{tikzpicture}[x=0.75pt,y=0.75pt,yscale=-1,xscale=1]
%uncomment if require: \path (0,300); %set diagram left start at 0, and has height of 300

%Shape: Triangle [id:dp053948984035182] 
\draw   (360,20) -- (390,70.33) -- (330,70.33) -- cycle ;
%Straight Lines [id:da38899678532814896] 
\draw [color={rgb, 255:red, 255; green, 0; blue, 0 }  ,draw opacity=1 ]   (345.33,44.33) -- (360.33,70.33) ;
%Shape: Triangle [id:dp9123444492379096] 
\draw   (260,20) -- (290,70.33) -- (230,70.33) -- cycle ;
%Straight Lines [id:da3333274202694039] 
\draw    (290.33,40.33) -- (328.33,40.33) ;
\draw [shift={(330.33,40.33)}, rotate = 180] [color={rgb, 255:red, 0; green, 0; blue, 0 }  ][line width=0.75]    (10.93,-3.29) .. controls (6.95,-1.4) and (3.31,-0.3) .. (0,0) .. controls (3.31,0.3) and (6.95,1.4) .. (10.93,3.29)   ;
%Shape: Rectangle [id:dp658152795456475] 
\draw   (121,100.33) -- (170.33,100.33) -- (170.33,150.33) -- (121,150.33) -- cycle ;
%Straight Lines [id:da4178479503741537] 
\draw [color={rgb, 255:red, 255; green, 0; blue, 0 }  ,draw opacity=1 ]   (231.33,125.33) -- (256.33,150.33) ;
%Shape: Rectangle [id:dp13970992998037612] 
\draw   (231,100.33) -- (280.33,100.33) -- (280.33,150.33) -- (231,150.33) -- cycle ;
%Straight Lines [id:da7168002284640573] 
\draw    (180.33,125.33) -- (218.33,125.33) ;
\draw [shift={(220.33,125.33)}, rotate = 180] [color={rgb, 255:red, 0; green, 0; blue, 0 }  ][line width=0.75]    (10.93,-3.29) .. controls (6.95,-1.4) and (3.31,-0.3) .. (0,0) .. controls (3.31,0.3) and (6.95,1.4) .. (10.93,3.29)   ;
%Straight Lines [id:da6092781758766023] 
\draw [color={rgb, 255:red, 255; green, 0; blue, 0 }  ,draw opacity=1 ]   (145.33,210.33) -- (145.33,260.33) ;
%Shape: Rectangle [id:dp4001001964953048] 
\draw   (121,210.33) -- (170.33,210.33) -- (170.33,260.33) -- (121,260.33) -- cycle ;
%Straight Lines [id:da7178391716405099] 
\draw    (145.33,160.33) -- (145.33,198.33) ;
\draw [shift={(145.33,200.33)}, rotate = 270] [color={rgb, 255:red, 0; green, 0; blue, 0 }  ][line width=0.75]    (10.93,-3.29) .. controls (6.95,-1.4) and (3.31,-0.3) .. (0,0) .. controls (3.31,0.3) and (6.95,1.4) .. (10.93,3.29)   ;
%Shape: Regular Polygon [id:dp33160480080690447] 
\draw   (360.29,100.34) -- (386.39,119.44) -- (376.29,150.16) -- (343.95,150.06) -- (334.06,119.26) -- cycle ;
%Straight Lines [id:da45618558829906886] 
\draw [color={rgb, 255:red, 255; green, 0; blue, 0 }  ,draw opacity=1 ]   (338.33,133.33) -- (360.33,150.33) ;
%Straight Lines [id:da6044104143635626] 
\draw [color={rgb, 255:red, 255; green, 0; blue, 0 }  ,draw opacity=1 ]   (492.67,219.33) -- (479.67,260.33) ;
%Straight Lines [id:da4256082202655984] 
\draw    (480.33,160) -- (480.33,198) ;
\draw [shift={(480.33,200)}, rotate = 270] [color={rgb, 255:red, 0; green, 0; blue, 0 }  ][line width=0.75]    (10.93,-3.29) .. controls (6.95,-1.4) and (3.31,-0.3) .. (0,0) .. controls (3.31,0.3) and (6.95,1.4) .. (10.93,3.29)   ;
%Straight Lines [id:da5844395653617523] 
\draw    (440.33,130.33) -- (402.33,130.33) ;
\draw [shift={(400.33,130.33)}, rotate = 360] [color={rgb, 255:red, 0; green, 0; blue, 0 }  ][line width=0.75]    (10.93,-3.29) .. controls (6.95,-1.4) and (3.31,-0.3) .. (0,0) .. controls (3.31,0.3) and (6.95,1.4) .. (10.93,3.29)   ;
%Shape: Regular Polygon [id:dp2820123358797324] 
\draw   (480.29,100.34) -- (506.39,119.44) -- (496.29,150.16) -- (463.95,150.06) -- (454.06,119.26) -- cycle ;
%Shape: Regular Polygon [id:dp4174131602657527] 
\draw   (480.29,210.34) -- (506.39,229.44) -- (496.29,260.16) -- (463.95,260.06) -- (454.06,229.26) -- cycle ;

\end{tikzpicture}
}
    \caption{Différentes manières de découper les polygones $Y_n=3,4,5$}
\end{figure}

\noindent Un triangle peut donner un triangle ou un quadrilatère avec probabilité
\[
    \mathbb{P}[Y_{n+1}=3\mid Y_n=3]=\mathbb{P}[Y_{n+1}=4\mid Y_n=3]=\frac{1}{2}
\]
De même, un quadrilatère peut donner un triangle, un quadrilatère ou un pentagone avec 
probabilité
\[
    \mathbb{P}[Y_{n+1}=3\mid Y_n=4]=\mathbb{P}[Y_{n+1}=4\mid Y_n=4]=\mathbb{P}[Y_{n+1}=5\mid Y_n=4]=\frac{1}{3}
\]
Par suite logique, on remarque que la probabilité dépend de l'état dans lequel nous 
sommes (donc dépend du nombre de côtés du polygone actuel) :
\[
    p_{i,j}=\mathbb{P}[X_{n+1}=j\mid X_n=i]=\frac{1}{i+2}
\]
Grâce à cette formule, on peut déduire à quoi va ressembler notre matrice de transitions.
À chaque fois que le polygone $i$ choisit augmente d'un côté de plus que son polygone
antécédent, il peut atteindre le dernier polygone et tous ceux qui l'ont antécéder avec
probabilité $1/(i+2)$.
\[
    \boldsymbol{P}=\left(
        \begingroup
        \renewcommand{\arraystretch}{1.3}
        \setlength{\arraycolsep}{3pt}
        \small
        \begin{array}{*{6}{c}}
        p_{0,0} & p_{0,1} & 0 & 0 & \cdots & 0 \\
        p_{1,0} & p_{1,1} & p_{1,2} & 0 & \cdots & 0 \\
        \vdots & \vdots & \vdots & \vdots & \ddots & \vdots \\
        p_{n,0} & p_{n,1} & p_{n,2} & p_{n,3} &  \cdots & p_{n,n}
    \end{array}
    \endgroup
    \right)
    =
    \left(
        \begingroup
        \renewcommand{\arraystretch}{1.3}
        \setlength{\arraycolsep}{2pt}
        \small
        \begin{array}{*{7}{c}}
        \frac{1}{2} & \frac{1}{2} & 0 & 0 & \cdots & 0 & 0 \\
        \frac{1}{3} & \frac{1}{3} & \frac{1}{3} & 0 & \cdots & 0 & 0 \\
        \vdots & \vdots & \vdots & \ddots & \cdots & \vdots & \vdots \\
        \frac{1}{i+2} & \frac{1}{i+2} & \frac{1}{i+2} & \frac{1}{i+2} & \frac{1}{i+2} & 0 & 0 \\
        \vdots & \vdots & \vdots & \vdots & \vdots & \ddots & \vdots \\
        \frac{1}{n+2} & \frac{1}{n+2} & \frac{1}{n+2} & \frac{1}{n+2} & \frac{1}{n+2} & \cdots & \frac{1}{n+2}
    \end{array}
    \endgroup
    \right)
\]
Notre chaîne de Markov $\{X_n,\,n\ge 0\}$ est donc irréductible parce qu'on peut partir de
n'importe quel état et se rendre à tout autre état (polygone) en $j-i$ étapes lorsqu'on part
de $i$ et on se rend à $j$ (avec $i<j$) et en 1 étape si on part de $j$ et on se rend à $i$ 
(encore où $i<j$).

\newpage
\subsection*{Partie B}
\textbf{Énoncé}. Montrez que les probabilités d'état stationnaire de cette chaîne sont
données par l'équation étonnamment élégante suivante : $\pi_i=1/(i!\cdot e)$ pour 
$i\ge0$, où $e=2.71828...$ est la constante d'Euler.
\vspace{.2cm}
\hrule
\vspace{.4cm}
Pour notre chaîne de Markov irréductible positive, les états stationnaires $\pi_i$ forment 
les équations d'équilibre
\begin{gather}
    \pi_j=\sum_{i\in\mathbb{N}}\pi_i\cdot p_{i,j},\,\forall j\in\mathbb{N} \\
    \sum_{i\in\mathbb{N}} \pi_i=1
\end{gather}
On peut donc retrouver l'équation de $\pi_0$ et ceux de $\pi_j$ avec
\[
    \pi_0=\sum_{i=0}^{\infty}\frac{\pi_i}{i+2}\ \text{et}\ \pi_j=\sum_{i=j-1}^{\infty}\frac{\pi_i}{i+2}
\]
Donc pour $j=1$ on retrouve
\[
    \pi_1=\sum_{i=0}^{\infty}\frac{\pi_i}{i+2}=\frac{\pi_0}{2}+\pi_2
\]
Nous remarquons également que 
\[
\pi_0=\sum_{i=0}^{\infty}\frac{\pi_i}{i+2}=\pi_1
\]
Nous allons maintenant examiner le cas $j=2$ :
\[
    \pi_2=\sum_{i=1}^{\infty}\frac{\pi_i}{i+2}=\frac{\pi_1}{3}+\sum_{i=2}^{\infty}\frac{\pi_i}{i+2}=\frac{\pi_1}{3}+\pi_3
\]
Ensuite, pour $j=3$ nous avons :
\[
    \pi_3=\sum_{i=2}^{\infty}\frac{\pi_i}{i+2}=\frac{\pi_2}{4}+\sum_{i=3}^{\infty}\frac{\pi_i}{i+2}=\frac{\pi_2}{4}+\pi_4
\]
À partir de ces relations, nous pouvons exprimer les états stationnaires $\pi_i$ en 
fonction de $\pi_0$ :
\begin{gather}
    \pi_2=\pi_1-\frac{\pi_0}{2}=\pi_0-\frac{\pi_0}{2}=\frac{\pi_0}{2} \notag \\
    \pi_3=\pi_2-\frac{\pi_1}{3}=\frac{\pi_0}{2}-\frac{\pi_0}{3}=\frac{\pi_0}{6} \notag \\
    \pi_4=\pi_3-\frac{\pi_2}{4}=\frac{\pi_0}{6}-\frac{\pi_0}{8}=\frac{\pi_0}{24} \notag
\end{gather}
Nous constatons ainsi une séquence logique où la probabilité stationnaire dépend de 
l'état considéré. Nous obtenons donc pour l'état stationnaire $j$ :
\begin{equation}
    \pi_j=\frac{\pi_0}{j!}
\end{equation}
Pour démontrer que cette équation est valide $\forall\, j$, nous allons démontrer qu'elle
est vraie pour $j+1$ en utilisant l'équation (13) utilisée plutôt pour trouver $\pi_1,\,\pi_2,\,etc$ :
\begin{align}
    \pi_j&=\frac{\pi_{j-1}}{j+1}+\pi_{j+1} \notag \\
    \pi_{j+1}&=\pi_j-\frac{\pi_{j-1}}{j+1} \notag \\
    &=\frac{\pi_0}{j!}-\frac{\pi_0}{(j-1)!\cdot (j+1)} \notag \\
    &=\frac{\pi_0}{(j-1)!}\left(\frac{1}{j}-\frac{1}{j+1}\right) \notag \\
    &=\frac{\pi_0}{(j-1)!}\left(\frac{(j+1)-j}{j(j+1)}\right) \notag \\
    &=\frac{\pi_0}{(j+1)\cdot (j)\cdot (j-1)!}=\frac{\pi_0}{(j+1)!}
\end{align}
Nous avons démontré que pour tout état $j$, les probabilités stationnaires peuvent s'exprimer 
sous la forme $\pi_j=\frac{\pi_0}{j!}$. Pour déterminer $\pi_0$, nous n'avons qu'à sommer sur les $\pi_j$ :
\[
    \sum_{\forall j}\pi_j=\pi_0\cdot\sum_{\forall j}\frac{1}{j!}
\]
On remarque qu'on a une série de Taylor de $e^x=\sum_{n=0}^{\infty}\frac{x^n}{n!}$ où 
$x=1$ dans le côté droit de l'égalité. Donc,
\[
    \sum_{\forall j}\pi_j=\pi_0\cdot e
\]
Nous avons démontré à la partie A que notre chaîne de Markov est irréductible. Cela 
signifie qu'il est possible de passer d'un état \( j \) à un autre état \( j+1 \) ou 
\( j-1 \), ce qui garantit une connexion entre tous les états de la chaîne (\( j-1 \leftrightarrow j \leftrightarrow j+1 \)). 
Le temps moyen de retour à l'état \( i \) est donc fini pour tous les états, 
ce qui renforce l'affirmation que la chaîne est récurrente positive. Nous savons également que lorsque la chaîne est récurrente positive, la somme des 
probabilités limites est égale à 1 :
\begin{gather}
    \pi_j=\lim_{n\to\infty}P_{i,j}^{(n)} \\
    \sum_{\forall j}\pi_j=\sum_{\forall j}\lim_{n\to\infty}P_{i,j}^{(n)}=1 \notag
\end{gather}
Donc nous avons
\[
    \sum_{\forall j}\pi_j=1=\pi_0\cdot e\to \pi_0=\frac{1}{e}
\]
La formule finale d'un état stationnaire $j$ est caractérisée par
\begin{equation}
    \pi_j=\frac{\pi_0}{j!}=\frac{1}{j!\cdot e},\, \forall\, j
\end{equation}


\end{document}

